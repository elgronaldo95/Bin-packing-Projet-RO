\documentclass{article}

\usepackage[utf8]{inputenc}
\usepackage{indentfirst}
\usepackage{listings}
\usepackage{fancyhdr}
\usepackage[a4paper, total={6in, 8in}]{geometry}

\title{Projet bin-packing - Recherche Opérationnelle}
\author{Dominik Tanko et Gloire Samba}
\date{2 Avril 2022}

\pagestyle{fancy}
\fancyhf{}
\rhead{684I}
\lhead{Projet bin-packing}
\cfoot{\thepage}

\begin{document}

\maketitle

\renewcommand*\contentsname{Sommaire}
\tableofcontents


\newpage

\section{Introduction}
    Le problème de bin-packing consiste à déterminer le nombre minimal de conteneurs (\textit{bins} en anglais) identiques pour ranger un ensemble d’objets de tailles diverses, sans qu’ils ne se chevauchent.
    Il nous est demandé de faire une implémentation des méthodes présentées dans le sujet du projet. Les méthodes présentées sont les suivantes : 
    \begin{itemize}
        \item modélisation directe
        \item modélisation indirecte uni-dimensionnel
        \item modélisation indirecte bi-dimensionnel
    \end{itemize}
    Pour faire cela nous utilisons Julia/JuMP/GLPK. Pour chaque modélisation de problème nous avons le choix entre différentes structures de données. Ces choix seront expliqués pour chaque section.

\section{Heuristique best-fit}
    Le principe d’une heuristique est de construire rapidement une solution admissible sans avoir nécessairement de garantie théorique sur la qualité de la solution obtenue. Ici nous avons deux étapes principales pour obtenir un nombre de bins "minimal".
    \begin{enumerate}
        \item On trie les objets par taille décroissante,
        \item On range chaque objet (dans l’ordre du tri) dans le bin ouvert le plus rempli, parmi les bins pouvant l’accueillir. Si aucun bin ouvert n’a la place suffisante, on ouvre un nouveau bin.
    \end{enumerate}
    Pour faire cela nous avons crée une fonction \texttt{heuristique\_best\_fit(d::données1D)}.
    Cette fonction prend en entré un struct \texttt{donnees1D} défini auparavant. \newline
    On commence par trier le table des objets stockés dans \texttt{d.tab}. On le trie en ordre décroissant avec l'algorithme \textit{Quicksort}. Nous avons choisi cet algorithme après avoir testé plusieurs algorithmes de tri \textit{(Merge sort, Bubble sort)}. La complexité temporelle de cet algorithme est de l'ordre $O(n^2)$ dans le pire cas et $O(n\log n)$ dans le cas moyen. La complexité spaciale est de l'ordre $O(n)$ dans le pire cas et $O(\log n)$ dans le cas moyen. \newline
    Après, nous parcourons \texttt{d.tab} et pour pour chaque objet nous avons une boucle qui va de 1 au nombre d'objet de cette taille. Une troisième boucle à l'intérieur essaie de trouver un bin où l'objet peut être placé. Si on trouve un tel bin, on diminue la taille disponible dans ce bin par la taille de l'objet. On continue d'itérer dans ces boucles jusqu'à nous n'avons plus d'objets à ranger. A la fin, nous retournons la taille de table des bins, qui nous donne le nombre de bin minimal nécessaire pour pouvoir ranger tous les objets.


\section{Modélisation directe}
    Pour résoudre ce problème, une modélisation nous est fournie, donc il s'agit donc de traduire cette modélisation au langage Julia. La seule fonction qu'on doit ajouter en plus est une fonction qui "transforme" \texttt{d.tab} en un tableau qui contient chaque objet réellement. C'est-à-dire que
    \begin{center}
        le tableau $[(5,2), (3,3)]$ devient $[5,5,3,3,3]$.
    \end{center}
    Cette fonction est \texttt{tab\_objet\_taille\_reelle(d::donnees1D)}, en utilisant cela nous pouvons traduire la modélisation facilement. \newline
    Si on essaie de résoudre les instances données dans \texttt{/1Dim/A} on peut voir qu'à part le fichier \texttt{jouet.dat} on n'arrive à rien résoudre. Le programme prend trop de temps. La même chose est observable avec les instances données dans \texttt{/1Dim/B}, le seul fichier résolu est \texttt{B4.dat}.

\section{Motifs}
    Avant de pouvoir commencer résoudre la modélisation indirecte, nous devons effectuer un traitement sur les données. Nous devons déterminer les différentes possibilités de remplissage maximum des bins. La fonction récursive \texttt{motifs\_calc()} est utilisée, ses entrées sont les suivantes :
    \begin{itemize}
        \item \texttt{indice\_type::Int} : utilisé pour parcourir les types possibles
        \item \texttt{somme::Int} : utilisé récursivement pour suivre la somme des types d'objets posés dans un bin
        \item \texttt{taille\_bin::Int} : la taille maximale d'un bin
        \item \texttt{motif\_local::Vector{Int}} : le tableau utilisé localement pour créer un bin
        \item \texttt{types::Vector{Int}} : le tableau des types possibles
        \item \texttt{motifs::Vector{Vector{Int}}} : le tableau des motifs finaux
    \end{itemize}
    Cette fonction utilise une boucle simple dans chaque itération récursive. On parcours le tableau \texttt{types} et pour chaque élément, nous vérifions qu'un \texttt{type} de données peut être placé dans un bin ou pas. Si c'est le cas, on ajoute ce \texttt{type} dans le tableau de \texttt{motif\_local}. Puis on fusionne ce \texttt{motif\_local} avec le tableau final des \texttt{motifs}. On doit également "nettoyer" ce \texttt{motif\_local} du dernier \texttt{type} ajouté, qui correspond à revenir un niveau dans l'arbre des \texttt{motifs}.
    

\section{Modélisation indirecte}
    \subsection*{Uni-dimensionnelle}
        Il s'agit de traduire la modélisation présentée dans le sujet, utilisant le pré-traitement des données avec la fonction présentée auparavant.

    \subsection*{Bi-dimensionnelle}
        Pour cette modélisation, nous pouvons encore reutiliser la fonction  \texttt{motifs\_calc}, le seul changement à effectuer est de changer la vérification si un objet peut rentrer dans un bin. On doit créer une fonction qui peut faire cette vérification.

\end{document}