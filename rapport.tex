\documentclass{article}
\usepackage[utf8]{inputenc}

\title{Projet bin-packing - Recherche Opérationnelle}
\author{Dominik Tanko et Gloire Samba}
\date{2 Avril 2022}

\begin{document}

\maketitle

\newpage

\section{Introduction}
    Le problème de bin-packing consiste à déterminer le nombre minimal de conteneurs (\textit{bins} en anglais) identiques pour ranger un ensemble d’objets de tailles diverses, sans qu’ils ne se chevauchent.
    Il nous est demandé de faire une implémentation des méthodes présentées dans le sujet du projet. Les méthodes présentées sont les suivantes : 
    \begin{itemize}
        \item heuristique best-fit
        \item modélisation directe
        \item modélisation indirecte uni-dimensionnel
        \item modélisation indirecte bi-dimensionnel
    \end{itemize}
    Pour faire cela nous utilisons Julia/JuMP/GLPK. Pour chaque modélisation de problème nous avons le choix entre différentes structures de données. Ces choix seront expliqués pour chaque section.

\section{Heuristique best-fit}
    Cette méthode consiste en triant les objets en ordre décroissant par rapport leur tailles

\section{Motifs}


\section{Conclusion}

\end{document}